\section{Your project idea}

In this section, first  redefine the problem in great details. Then describe your solutions. Try to use as many illustrations such as pictures, graphs, pseudo code as you see fit. Here are some examples of including pictures, graphs, and pseudo code in latex. 

%%%%%%%%%%%%%%%%%%%%%%%%%%%%%%%%%%%%%%%%%%%%%%%%%%%%%%%%%%%%%%%%%%%
%
% Commands to include a figure:
%
%%%%%%%%%%%%%%%%%%%%%%%%%%%%%%%%%%%%%%%%%%%%%%%%%%%%%%%%%%%%%%%%%%%
Figure~\ref{fig:fig1} shows how to include a picture in a Latex document. 
\begin{figure}[!ht]
 \centering
\includegraphics[width=3in]{sample_1}
\caption{\label{fig:fig1}This is my great security design.}
\end{figure}

%%%%%%%%%%%%%%%%%%%%%%%%%%%%%%%%%%%%%%%%%%%%%%%%%%%%%%%%%%%%%%%%%%%
%
% The following pseudo code is copied from 
% http://users.sdsc.edu/~ssmallen/latex/pseudocode.html
%
%%%%%%%%%%%%%%%%%%%%%%%%%%%%%%%%%%%%%%%%%%%%%%%%%%%%%%%%%%%%%%%%%%%
Figure~\ref{fig:fig2} is a sample pseudo code copied from this site \cite{pseudocode}. 

\begin{figure}[!ht]
 \centering

  \begin{pseudocode}[framebox]{reduce}{projection, x, y, f}
    \FOR i \GETS 1 \TO y/f \DO
      \BEGIN
      \FOR j \GETS 1 \TO x/f \DO
        \BEGIN
        sum \GETS 0 \\
        \FOR m \GETS 1 \TO f \DO
          \BEGIN
          \FOR n \GETS 1 \TO f \DO
            sum = sum + projection[i*f+m][j * f+n]  \\
          \END \\
        reducedProjection[i][j] = sum / (f * f) \\
        \END
      \END \\
  \RETURN{reducedProjection}
  \end{pseudocode}
  
  \caption{\label{fig:fig2}This is my great pseudo code}
  
  \end{figure}
  
%%%%%%%%%%%%%%%%%%%%%%%%%%%%%%%%%%%%%%%%%%%%%%%%%%%%%%%%%%%%%%%%%%%
%
% Latex float chart example
% Copied from http://www.texample.net/tikz/examples/simple-flow-chart/
%
%%%%%%%%%%%%%%%%%%%%%%%%%%%%%%%%%%%%%%%%%%%%%%%%%%%%%%%%%%%%%%%%%%%

Figure~\ref{fig:fig3} is a sample float chart is copied from this website \cite{floatchart}

\tikzstyle{decision} = [diamond, draw, fill=blue!20, 
    text width=4.5em, text badly centered, node distance=3cm, inner sep=0pt]
\tikzstyle{block} = [rectangle, draw, fill=blue!20, 
    text width=5em, text centered, rounded corners, minimum height=4em]
\tikzstyle{line} = [draw, -latex']
\tikzstyle{cloud} = [draw, ellipse,fill=red!20, node distance=3cm,
    minimum height=2em]
   
  \begin{figure}[!ht]
 \centering
 
\begin{tikzpicture}[node distance = 2cm, auto]
    % Place nodes
    \node [block] (init) {initialize model};
    \node [cloud, left of=init] (expert) {expert};
    \node [cloud, right of=init] (system) {system};
    \node [block, below of=init] (identify) {identify candidate models};
    \node [block, below of=identify] (evaluate) {evaluate candidate models};
    \node [block, left of=evaluate, node distance=3cm] (update) {update model};
    \node [decision, below of=evaluate] (decide) {is best candidate better?};
    \node [block, below of=decide, node distance=3cm] (stop) {stop};
    % Draw edges
    \path [line] (init) -- (identify);
    \path [line] (identify) -- (evaluate);
    \path [line] (evaluate) -- (decide);
    \path [line] (decide) -| node [near start] {yes} (update);
    \path [line] (update) |- (identify);
    \path [line] (decide) -- node {no}(stop);
    \path [line,dashed] (expert) -- (init);
    \path [line,dashed] (system) -- (init);
    \path [line,dashed] (system) |- (evaluate);
\end{tikzpicture}

  \caption{\label{fig:fig3}This is my great float chart}
  
  \end{figure}


%%%%%%%%%%%%%%%%%%%%%%%%%%%%%%%%%%%%%%%%%%%%%%%%%%%%%%%%%%%%%%%%%%%

\subsection{Mathematics}

If your project idea needs mathematics to formulate your methods, \LaTeX{} is great at typesetting mathematics. Let $X_1, X_2, \ldots, X_n$ be a sequence of independent and identically distributed random variables with $\text{E}[X_i] = \mu$ and $\text{Var}[X_i] = \sigma^2 < \infty$, and let
$$S_n = \frac{X_1 + X_2 + \cdots + X_n}{n}
      = \frac{1}{n}\sum_{i}^{n} X_i$$
denote their mean. Then as $n$ approaches infinity, the random variables $\sqrt{n}(S_n - \mu)$ converge in distribution to a normal $\mathcal{N}(0, \sigma^2)$.
