\documentclass[journal,twoside,web]{ieeecolor}
\usepackage{generic}
\usepackage{cite}
\usepackage{amsmath,amssymb,amsfonts}
\usepackage{algorithmic}
\usepackage{graphicx}
\usepackage{textcomp}
\def\BibTeX{{\rm B\kern-.05em{\sc i\kern-.025em b}\kern-.08em
    T\kern-.1667em\lower.7ex\hbox{E}\kern-.125emX}}
\markboth{\journalname, VOL. XX, NO. XX, XXXX 2017}
{Author \MakeLowercase{\textit{et al.}}: Preparation of Papers for IEEE TRANSACTIONS and JOURNALS (February 2017)}
\begin{document}
\title{Santander Customer Transaction Prediction (March 2019)}\\

\author{Adam Kulas, Ammar Ahmed, and Richard Ozara\\
\institute University of Waterloo,ON N2L 3G1, Canada\\akulas@uwaterloo.ca\\
aaafify@uwaterloo.ca \\\centerline{roozara@uwaterloo.ca}}

\maketitle



\begin{abstract}
These instructions give you guidelines for preparing papers for 
IEEE Transactions and Journals. Use this document as a template if you are 
using \LaTeX. Otherwise, use this document as an 
instruction set. The electronic file of your paper will be formatted further 
at IEEE. Paper titles should be written in uppercase and lowercase letters, 
not all uppercase. Avoid writing long formulas with subscripts in the title; 
short formulas that identify the elements are fine (e.g., "Nd--Fe--B"). Do 
not write ``(Invited)'' in the title. Full names of authors are preferred in 
the author field, but are not required. Put a space between authors' 
initials. The abstract must be a concise yet comprehensive reflection of 
what is in your article. In particular, the abstract must be self-contained, 
without abbreviations, footnotes, or references. It should be a microcosm of 
the full article. The abstract must be between 150--250 words. Be sure that 
you adhere to these limits; otherwise, you will need to edit your abstract 
accordingly. The abstract must be written as one paragraph, and should not 
contain displayed mathematical equations or tabular material. The abstract 
should include three or four different keywords or phrases, as this will 
help readers to find it. It is important to avoid over-repetition of such 
phrases as this can result in a page being rejected by search engines. 
Ensure that your abstract reads well and is grammatically correct.
\end{abstract}

\begin{IEEEkeywords}
Enter key words or phrases in alphabetical 
order, separated by commas. For a list of suggested keywords, send a blank 

\end{IEEEkeywords}

\section{Introduction}
\label{sec:introduction}
\IEEEPARstart{T}{his} document is a template for \LaTeX. If you  

\subsection{Abbreviations and Acronyms}


\subsection{Description of the data}



\subsection{Survey review on the dataset}




\subsection{Challenges in the dataset}


\subsection{Project Goal}




\section{Used Methods in the Project}


\subsection{Naive Bayes}



\subsection{Logistic Regression}


\subsection{XGBOOST}


\subsection{LGBM}

\section{Preprocessing}






\subsection{Resolution }
The proper resolution of your figures will depend on the type of figure it 
is as defined in the ``Types of Figures'' section. Author photographs, 
color, and grayscale figures should be at least 300dpi. Line art, including 
tables should be a minimum of 600dpi.

\subsection{Vector Art}
In order to preserve the figures' integrity across multiple computer 
platforms, we accept files in the following formats: .EPS/.PDF/.PS. All 
fonts must be embedded or text converted to outlines in order to achieve the 
best-quality results.

\subsection{Color Space}
The term color space refers to the entire sum of colors that can be 
represented within the said medium. For our purposes, the three main color 
spaces are Grayscale, RGB (red/green/blue) and CMYK 
(cyan/magenta/yellow/black). RGB is generally used with on-screen graphics, 
whereas CMYK is used for printing purposes.

All color figures should be generated in RGB or CMYK color space. Grayscale 
images should be submitted in Grayscale color space. Line art may be 
provided in grayscale OR bitmap colorspace. Note that ``bitmap colorspace'' 
and ``bitmap file format'' are not the same thing. When bitmap color space 
is selected, .TIF/.TIFF/.PNG are the recommended file formats.

\subsection{Accepted Fonts Within Figures}
When preparing your graphics IEEE suggests that you use of one of the 
following Open Type fonts: Times New Roman, Helvetica, Arial, Cambria, and 
Symbol. If you are supplying EPS, PS, or PDF files all fonts must be 
embedded. Some fonts may only be native to your operating system; without 
the fonts embedded, parts of the graphic may be distorted or missing.

A safe option when finalizing your figures is to strip out the fonts before 
you save the files, creating ``outline'' type. This converts fonts to 
artwork what will appear uniformly on any screen.

. 



\subsection{Referencing a Figure or Table Within Your Paper}
When referencing your figures and tables within your paper, use the 
abbreviation ``Fig.'' even at the beginning of a sentence. Do not abbreviate 
``Table.'' Tables should be numbered with Roman Numerals.

\subsection{Checking Your Figures: The IEEE Graphics Analyzer}
The IEEE Graphics Analyzer enables authors to pre-screen their graphics for 
compliance with IEEE Transactions and Journals standards before submission. 
The online tool, located at
\underline{http://graphicsqc.ieee.org/}, allows authors to 
upload their graphics in order to check that each file is the correct file 
format, resolution, size and colorspace; that no fonts are missing or 
corrupt; that figures are not compiled in layers or have transparency, and 
that they are named according to the IEEE Transactions and Journals naming 
convention. At the end of this automated process, authors are provided with 
a detailed report on each graphic within the web applet, as well as by 
email.

For more information on using the Graphics Analyzer or any other graphics 
related topic, contact the IEEE Graphics Help Desk by e-mail at 
graphics@ieee.org.

\subsection{Submitting Your Graphics}
Because IEEE will do the final formatting of your paper,
you do not need to position figures and tables at the top and bottom of each 
column. In fact, all figures, figure captions, and tables can be placed at 
the end of your paper. In addition to, or even in lieu of submitting figures 
within your final manuscript, figures should be submitted individually, 
separate from the manuscript in one of the file formats listed above in 


\subsection{Color Processing/Printing in IEEE Journals}
All IEEE Transactions, Journals, and Letters allow an author to publish 
color figures on IEEE Xplore\textregistered\ at no charge, and automatically 
convert them to grayscale for print versions. In most journals, figures and 
tables may alternatively be printed in color if an author chooses to do so. 
Please note that this service comes at an extra expense to the author. If 
you intend to have print color graphics, include a note with your final 
paper indicating which figures or tables you would like to be handled that 
way, and stating that you are willing to pay the additional fee.

\section{Conclusion}
A conclusion section is not required. Although a conclusion may review the 
main points of the paper, do not replicate the abstract as the conclusion. A 
conclusion might elaborate on the importance of the work or suggest 
applications and extensions. 

\appendices

Appendixes, if needed, appear before the acknowledgment.

\section*{Acknowledgment}



\section*{References and Footnotes}

\subsection{References}






\begin{thebibliography}{00}

\bibitem{b1} G. O. Young, ``Synthetic structure of industrial plastics,'' in \emph{Plastics,} 2\textsuperscript{nd} ed., vol. 3, J. Peters, Ed. New York, NY, USA: McGraw-Hill, 1964, pp. 15--64.

\bibitem{b2} W.-K. Chen, \emph{Linear Networks and Systems.} Belmont, CA, USA: Wadsworth, 1993, pp. 123--135.

\bibitem{b3} J. U. Duncombe, ``Infrared navigation---Part I: An assessment of feasibility,'' \emph{IEEE Trans. Electron Devices}, vol. ED-11, no. 1, pp. 34--39, Jan. 1959, 10.1109/TED.2016.2628402.

\bibitem{b4} E. P. Wigner, ``Theory of traveling-wave optical laser,'' \emph{Phys. Rev}., vol. 134, pp. A635--A646, Dec. 1965.

\bibitem{b5} E. H. Miller, ``A note on reflector arrays,'' \emph{IEEE Trans. Antennas Propagat}., to be published.

\bibitem{b6} E. E. Reber, R. L. Michell, and C. J. Carter, ``Oxygen absorption in the earth's atmosphere,'' Aerospace Corp., Los Angeles, CA, USA, Tech. Rep. TR-0200 (4230-46)-3, Nov. 1988.

\bibitem{b7} J. H. Davis and J. R. Cogdell, ``Calibration program for the 16-foot antenna,'' Elect. Eng. Res. Lab., Univ. Texas, Austin, TX, USA, Tech. Memo. NGL-006-69-3, Nov. 15, 1987.

\bibitem{b8} \emph{Transmission Systems for Communications}, 3\textsuperscript{rd} ed., Western Electric Co., Winston-Salem, NC, USA, 1985, pp. 44--60.

\bibitem{b9} \emph{Motorola Semiconductor Data Manual}, Motorola Semiconductor Products Inc., Phoenix, AZ, USA, 1989.

\bibitem{b10} G. O. Young, ``Synthetic structure of industrial
plastics,'' in Plastics, vol. 3, Polymers of Hexadromicon, J. Peters,
Ed., 2\textsuperscript{nd} ed. New York, NY, USA: McGraw-Hill, 1964, pp. 15-64.
[Online]. Available:
\underline{http://www.bookref.com}.



\end{thebibliography}






\end{document}
